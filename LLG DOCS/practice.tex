\documentclass[12pt,letterpaper, onecolumn]{exam}
\usepackage{amsmath}
\usepackage{amssymb}
\usepackage[lmargin=71pt, tmargin=1.2in]{geometry} 
\lhead{Chapitre \thesection\\}
\rhead{It's a me, yassine \\}
\thispagestyle{empty}
\setlength{\parindent}{0pt}
\usepackage{tikz}

\newcommand*\circled[1]{\tikz[baseline=(char.base)]{
            \node[shape=circle,draw,inner sep=2pt] (char) {#1};}}

\begin{document}

\begingroup  
    \centering
    \LARGE Transition CPGE\\
\endgroup
\rule{\textwidth}{0.4pt}
\pointsdroppedatright
\printanswers
\renewcommand{\solutiontitle}[1]{\noindent\textbf{}\enspace} 

\newcommand{\answer}[1]{\begin{solution} #1 \end{solution}}

\begin{questions}

\section*{Chapitre 0 : Outils latex}

\question*[1] Calculer 1+1 \circled{0} :
    
\answer{La solution est triviale, on a 1+1 = 2.}
        
\newpage
 
 
 \section{Chapitre 1 :  Rédaction, modes de raisonnements}
 \setcounter{question}{0}
 \question[1] \circled{2} (Somme des cubes des $n$ premiers cubes*) Montrer que : 
 $$\sum_{k=1}^n k^3  = \left (\frac{n(n+1)}{2} \right)^2$$
 \answer{
On pose la propriété $P_n : \displaystyle \sum_{k=1}^n k^3  = \left (\frac{n(n+1)}{2} \right)^2$. \\
$P_1$ est naturellement vérifiée.
Supposons que $P_n$ est vraie :
$$\sum_{k=1}^n k^3  = \left (\frac{n(n+1)}{2} \right)^2$$
$$\iff  \left (\sum_{k=1}^n k^3 \right)  +(n+1)^3=\frac{n^2(n+1)^2}{4}  + (n+1)^3 = \frac{n^2(n+1)^2}{4}  + (n^3+3n^2+3n+1)$$
$$\iff  \sum_{k=1}^{n+1} k^3 = \frac{n^2(n^2 + 2n + 1) + 4n^3 +12n^2 + 12n+4)}{4} $$
$$\iff  \sum_{k=1}^{n+1} k^3=  \frac{n^4 + 2n^3 + n^2 + 4n^3 }{4}$$
$$\iff  \sum_{k=1}^{n+1} k^3=  \frac{n^4 + 6n^3 + 5n^2 + 12n + 4}{4}$$
Or, d'après notre propriété, le numérateur vaudrait :  $$ ((n+1)(n+2))^2 = n^4 + 6n^3+ 5n^2 + 12n + 4$$
D'où la propriété $P_n$ est vraie pour tout $n \in \mathbb{N}.$
 }
 
\question[1] \circled{2} Montrer que, si $n \in \mathbb{N}$, il existe un entier impair $\lambda_n$ tel que :
$$5^{2^n} = 1 + \lambda_n2^{n+2}$$
 
 \answer{
 Soit $P_n : \forall n, n \in \mathbb{N}, \exists \lambda_n : 5^{2^n} = 1 + \lambda_n 2^{n+2}$ .\\
 Pour $P_1 : $ On a $5^2 = 25 = 1 + (3) \cdot 2^3$. \\
D'où $\lambda_1 = 3 : $ L'initialisation est vérifiée.

Supposons $P_n$ vraie : 
 $$5^{2^n} = 1 + \lambda_n 2^{n+2} \iff 5^2 \cdot \left ( 5^{2^n}\right) = 5^2 \cdot \left (1 + \lambda_n 2^{n+2} \right)$$
 $$\iff 5^{2^{n+1}} = 25 + 25\lambda_n 2^{n+2}$$
 Donc on aurait $\lambda_{n+1} = 24 + $
 }
 
\end{questions}
\end{document}